\chapter{Proof Systems}

\begin{definition}[Proof System]
    A \Keyword{proof system} is used to determine the \textit{validity} of formulas.
\end{definition}

\begin{remark}
    A trivial system would be to generate the truth table for some formula $\phi$ and
    check all rows yield $\top$. But for $n$ variables we will need $2^n$ rows which
    explodes exponentially.
    
    Thus, we need another system, which analyzes the syntax of the formula to
    determine its validity.
    
    Changes on syntactical level may not be semantics-preserving. To ensure that
    syntactical transformations are semantics-preserving, it is critical that the
    proof system used is both \Keyword{sound} and \Keyword{complete}.
    
    Let
    \begin{itemize}
        \item $\models \phi$ \quad denote $\phi$ is \textit{valid}.
        \item $\vdash \phi$ \quad denote there exists a \textit{proof} of $\phi$.
    \end{itemize}
\end{remark}

\section{Soundness and Completeness}

\begin{definition}[Soundness]
    A proof system is \Keyword{sound} iff it can \textit{only} prove \textit{valid} formulas.
    
    \begin{equation}
        \vdash \phi \implies \models \phi
    \end{equation}
\end{definition}

\begin{definition}[Completeness]
    If a formula is \textit{valid}, then the proof system \textit{can} prove it.
    
    \begin{equation}
        \models \phi \implies \vdash \phi
    \end{equation}
\end{definition}

\section{Axiomatic Proof Systems}

Let us define a simple language \textsc{ImpNeg} based on a subset of propositional language.

\begin{definition}[\textsc{ImpNeg}]
    Let \textsc{ImpNeg} be a propositional language with only
    
    \begin{enumerate}
        \item \Keyword{Implication} $\to$; and
        \item \Keyword{Negation} $\neg$.
    \end{enumerate}
\end{definition}

\begin{definition}[\textsc{ImpNeg} Axiom Schemas]\label{axi_sch:imp_neg}
    From the definition of \textsc{ImgNeg}, we can derive three axiom schemas:
    
    \begin{itemize}
        \item \Keyword{I}: $(\phi \to (\psi \to \phi))$.
        \item \Keyword{II}: $((\phi \to (\psi \to \kappa)) \to 
            ((\phi \to \psi) \to (\phi \to \kappa)))$.
        \item \Keyword{III}: $((\neg \phi \to \neg \psi) \to (\psi \to \phi))$.
    \end{itemize}
\end{definition}

\begin{definition}[Axiom]
    An \Keyword{axiom} can be obtained from \textit{substituting} any formulas
    in the places of $\phi$, $\psi$ and $\kappa$ from the schemas.
\end{definition}

\begin{definition}[Inference Rule]
    An \Keyword{inference rule} allows us to obtain a \Keyword{conclusion} from
    $n$ number of \Keyword{premises}; it has the form
    
    \begin{figure}[H]
        \centering
        \begin{prooftree}
            \AxiomC{$\phi_1$}
            \AxiomC{$\phi_2$}
            \AxiomC{$\ldots$}
            \AxiomC{$\phi_n$}
            \RightLabel{(Name of Inference Rule)}
            \QuaternaryInfC{$\psi$}
        \end{prooftree}
        \caption{Form of Inference Rules. Any $\phi_i$ is a \textit{premise} and $\psi$ 
            is the \textit{conclusion}.}
        \label{fig:inference_rule}
    \end{figure}
\end{definition}

\begin{definition}[Modus Ponens]
    A typical \textit{inference rule} is \Keyword{modus ponens}, given by
    
    \begin{figure}[H]
        \centering
        \begin{prooftree}
            \AxiomC{$\phi$}
            \AxiomC{$(\phi \to \psi)$}
            \RightLabel{(Modus Ponens)}
            \BinaryInfC{$\psi$}
        \end{prooftree}
        \caption{Modus Ponens. When both $\phi$ and $(\phi \to \psi)$ has been proved
            then $\psi$ can be derived.}
        \label{fig:modus_ponens}
    \end{figure}
\end{definition}

\begin{definition}[Proof]
    A \Keyword{proof} is then a \textit{sequence} of formulas
    
    \begin{equation}
        \phi_0, \phi_1, \ldots, \phi_n
    \end{equation}
    
    Such that for formulas $\phi_i$ with $i \le n$, $\phi_i$ is either:
    \begin{enumerate}
        \item An instance of an \textit{axiom}; or
        \item Derived by \textit{modus ponens} from $\phi_j$, $\phi_k$ for some $j, k < i$.
            \begin{itemize}
                \item For instance, if $\phi_k = (\phi_j \to \phi_i)$.
            \end{itemize}
    \end{enumerate}
    
    Given that such proof exists, then $\phi_n$ is a \Keyword{theorem} and may be denoted
    as
    \begin{equation}
        \vdash \phi_n
    \end{equation}
\end{definition}
