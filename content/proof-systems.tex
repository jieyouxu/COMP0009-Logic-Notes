\chapter{Proof Systems}

\begin{definition}[Proof System]
    A \Keyword{proof system} is used to determine the \textit{validity} of formulas.
\end{definition}

\begin{remark}
    A trivial system would be to generate the truth table for some formula $\phi$ and
    check all rows yield $\top$. But for $n$ variables we will need $2^n$ rows which
    explodes exponentially.
    
    Thus, we need another system, which analyzes the syntax of the formula to
    determine its validity.
    
    Changes on syntactical level may not be semantics-preserving. To ensure that
    syntactical transformations are semantics-preserving, it is critical that the
    proof system used is both \Keyword{sound} and \Keyword{complete}.
    
    Let
    \begin{itemize}
        \item $\models \phi$ \quad denote $\phi$ is \textit{valid}.
        \item $\vdash \phi$ \quad denote there exists a \textit{proof} of $\phi$.
    \end{itemize}
\end{remark}

\section{Soundness and Completeness}

\begin{definition}[Soundness]
    A proof system is \Keyword{sound} iff it can \textit{only} prove \textit{valid} formulas.
    
    \begin{equation}
        \vdash \phi \implies \models \phi
    \end{equation}
\end{definition}

\begin{definition}[Completeness]
    If a formula is \textit{valid}, then the proof system \textit{can} prove it.
    
    \begin{equation}
        \models \phi \implies \vdash \phi
    \end{equation}
\end{definition}

\section{Axiomatic Proof Systems}

Let us define a simple language \textsc{ImpNeg} based on a subset of propositional language.

\begin{definition}[\textsc{ImpNeg}]
    Let \textsc{ImpNeg} be a propositional language with only
    
    \begin{enumerate}
        \item \Keyword{Implication} $\to$; and
        \item \Keyword{Negation} $\neg$.
    \end{enumerate}
\end{definition}

\begin{definition}[\textsc{ImpNeg} Axiom Schemas]\label{axi_sch:imp_neg}
    From the definition of \textsc{ImgNeg}, we can derive three axiom schemas:
    
    \begin{itemize}
        \item \Keyword{I}: $(\phi \to (\psi \to \phi))$.
        \item \Keyword{II}: $((\phi \to (\psi \to \kappa)) \to 
            ((\phi \to \psi) \to (\phi \to \kappa)))$.
        \item \Keyword{III}: $((\neg \phi \to \neg \psi) \to (\psi \to \phi))$.
    \end{itemize}
\end{definition}

\begin{definition}[Axiom]
    An \Keyword{axiom} can be obtained from \textit{substituting} any formulas
    in the places of $\phi$, $\psi$ and $\kappa$ from the schemas.
\end{definition}

\begin{definition}[Inference Rule]
    An \Keyword{inference rule} allows us to obtain a \Keyword{conclusion} from
    $n$ number of \Keyword{premises}; it has the form
    
    \begin{figure}[H]
        \centering
        \begin{prooftree}
            \AxiomC{$\phi_1$}
            \AxiomC{$\phi_2$}
            \AxiomC{$\ldots$}
            \AxiomC{$\phi_n$}
            \RightLabel{(Name of Inference Rule)}
            \QuaternaryInfC{$\psi$}
        \end{prooftree}
        \caption{Form of Inference Rules. Any $\phi_i$ is a \textit{premise} and $\psi$ 
            is the \textit{conclusion}.}
        \label{fig:inference_rule}
    \end{figure}
\end{definition}

\begin{definition}[Modus Ponens]
    A typical \textit{inference rule} is \Keyword{modus ponens}, given by
    
    \begin{figure}[H]
        \centering
        \begin{prooftree}
            \AxiomC{$\phi$}
            \AxiomC{$(\phi \to \psi)$}
            \RightLabel{(Modus Ponens)}
            \BinaryInfC{$\psi$}
        \end{prooftree}
        \caption{Modus Ponens. When both $\phi$ and $(\phi \to \psi)$ has been proved
            then $\psi$ can be derived.}
        \label{fig:modus_ponens}
    \end{figure}
\end{definition}

\begin{definition}[Proof]
    A \Keyword{proof} is then a \textit{sequence} of formulas
    
    \begin{equation}
        \phi_0, \phi_1, \ldots, \phi_n
    \end{equation}
    
    Such that for formulas $\phi_i$ with $i \le n$, $\phi_i$ is either:
    \begin{enumerate}
        \item An instance of an \textit{axiom}; or
        \item Derived by \textit{modus ponens} from $\phi_j$, $\phi_k$ for some $j, k < i$.
            \begin{itemize}
                \item For instance, if $\phi_k = (\phi_j \to \phi_i)$.
            \end{itemize}
    \end{enumerate}
    
    Given that such proof exists, then $\phi_n$ is a \Keyword{theorem} and may be denoted
    as
    \begin{equation}
        \vdash \phi_n
    \end{equation}
\end{definition}

\begin{example}
    Let us try to prove $\vdash (p \to p)$ with the axiom schema.
    
    \begin{align*}
        &((p \to ((p \to p) \to p)) \to ((p \to (p \to p)) \to (p \to p)))
            &\quad\text{ (1) (Axiom $\mathbf{II}$) } \\
        &(p \to ((p \to p) \to p))
            &\quad\text{ (2) (Axiom $\mathbf{I}$) } \\
        &((p \to (p \to p)) \to (p \to p))
            &\quad\text{ (3) (M.P. from (1) and (2)) } \\
        &(p \to (p \to p))
            &\quad\text{ (4) (Axiom $\mathbf{I}$) } \\
        &(p \to p)
            &\quad\text{ (5) (M.P. from (3) and (4)) }
    \end{align*}
\end{example}

\begin{definition}[Axiom Schemas for Propositional Logic]\label{axi_sch:prop_logic}
    To extend \textsc{ImpNeg} to cover propositional language, we add new axiom schemas
    to that of Definition \ref{axi_sch:prop_logic} to handle \textit{double negation}, 
    \textit{disjunction} and \textit{conjunction}.
    
    \begin{enumerate}
        \item \Keyword{IV}: $(p \to \neg \neg p)$ and $(\neg \neg p \to q)$
        \item \Keyword{V}: $((p \lor q) \Iff (\neg p \to q))$
        \item \Keyword{VI}: $((p \land q) \Iff \neg(p \to \neg q)$
    \end{enumerate}
\end{definition}

\begin{definition}[Proof with Assumptions]
    Let $\Gamma$ denote the set of \Keyword{assumptions}.
    
    Then the existence of a proof of $\phi$ using \textit{assumptions} from $\Gamma$
    is denoted as
    
    \begin{equation}
        \Gamma \vdash \phi
    \end{equation}
    
    In particular, there is a sequence
    
    \begin{equation}
        \phi_0, \phi_1, \ldots, \phi_n    
    \end{equation}
    
    Where $\phi = \phi_n$ and each of $\phi_i$ given $i \le n$ is either:
    
    \begin{enumerate}
        \item An instance of an \textit{axiom schema}; or
        \item An \textit{assumption} $\phi_i \in \Gamma$; or
        \item Obtained from $\phi_j, \phi_k$ given $j, k < i$ via \textit{modus ponens}.
    \end{enumerate}
\end{definition}

\subsection{Soundness, Completeness and Termination}

To check \Keyword{soundness} of axiomatic proof systems, we check that:

\begin{enumerate}
    \item All \textit{axioms} are \textit{valid}.
        \begin{itemize}
            \item Check that the following holds:
                \begin{prooftree}
                    \AxiomC{$\phi$}
                    \AxiomC{$(\phi \to \psi)$}
                    \RightLabel{(Modus Ponens)}
                    \BinaryInfC{$\psi$}
                \end{prooftree}
        \end{itemize}
    \item Deduce that all \textit{provable} formulas are valid by \textit{induction}
        on the \textit{length} of proofs, such that
        \begin{equation}
            \vdash \phi \implies \models \phi
        \end{equation}
\end{enumerate}

To check \Keyword{completeness} fo axiomatic proof systems, we need to check
that:

\begin{equation}
    \models \phi \implies \vdash \phi
\end{equation}

\begin{remark}
    Note that axiomatic proof systems are \textit{not} guaranteed to \textit{terminate}.
\end{remark}
