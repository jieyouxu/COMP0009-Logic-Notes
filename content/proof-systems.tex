\chapter{Proof Systems}

\begin{definition}[Proof System]
    A \Keyword{proof system} is used to determine the \textit{validity} of formulas.
\end{definition}

\begin{remark}
    A trivial system would be to generate the truth table for some formula $\phi$ and
    check all rows yield $\top$. But for $n$ variables we will need $2^n$ rows which
    explodes exponentially.
    
    Thus, we need another system, which analyzes the syntax of the formula to
    determine its validity.
    
    Changes on syntactical level may not be semantics-preserving. To ensure that
    syntactical transformations are semantics-preserving, it is critical that the
    proof system used is both \Keyword{sound} and \Keyword{complete}.
    
    Let
    \begin{itemize}
        \item $\models \phi$ \quad denote $\phi$ is \textit{valid}.
        \item $\vdash \phi$ \quad denote there exists a \textit{proof} of $\phi$.
    \end{itemize}
\end{remark}

\section{Soundness and Completeness}

\begin{definition}[Soundness]
    A proof system is \Keyword{sound} iff it can \textit{only} prove \textit{valid} formulas.
    
    \begin{equation}
        \vdash \phi \implies \models \phi
    \end{equation}
\end{definition}

\begin{definition}[Completeness]
    If a formula is \textit{valid}, then the proof system \textit{can} prove it.
    
    \begin{equation}
        \models \phi \implies \vdash \phi
    \end{equation}
\end{definition}

\section{Axiomatic Proof Systems}

Let us define a simple language \textsc{ImpNeg} based on a subset of propositional language.

\begin{definition}[\textsc{ImpNeg}]
    Let \textsc{ImpNeg} be a propositional language with only
    
    \begin{enumerate}
        \item \Keyword{Implication} $\to$; and
        \item \Keyword{Negation} $\neg$.
    \end{enumerate}
\end{definition}

\begin{definition}[\textsc{ImpNeg} Axiom Schemas]\label{axi_sch:imp_neg}
    From the definition of \textsc{ImgNeg}, we can derive three axiom schemas:
    
    \begin{itemize}
        \item \Keyword{I}: $(\phi \to (\psi \to \phi))$.
        \item \Keyword{II}: $((\phi \to (\psi \to \kappa)) \to 
            ((\phi \to \psi) \to (\phi \to \kappa)))$.
        \item \Keyword{III}: $((\neg \phi \to \neg \psi) \to (\psi \to \phi))$.
    \end{itemize}
\end{definition}

\begin{definition}[Axiom]
    An \Keyword{axiom} can be obtained from \textit{substituting} any formulas
    in the places of $\phi$, $\psi$ and $\kappa$ from the schemas.
\end{definition}

\begin{definition}[Inference Rule]
    An \Keyword{inference rule} allows us to obtain a \Keyword{conclusion} from
    $n$ number of \Keyword{premises}; it has the form
    
    \begin{figure}[H]
        \centering
        \begin{prooftree}
            \AxiomC{$\phi_1$}
            \AxiomC{$\phi_2$}
            \AxiomC{$\ldots$}
            \AxiomC{$\phi_n$}
            \RightLabel{(Name of Inference Rule)}
            \QuaternaryInfC{$\psi$}
        \end{prooftree}
        \caption{Form of Inference Rules. Any $\phi_i$ is a \textit{premise} and $\psi$ 
            is the \textit{conclusion}.}
        \label{fig:inference_rule}
    \end{figure}
\end{definition}

\begin{definition}[Modus Ponens]
    A typical \textit{inference rule} is \Keyword{modus ponens}, given by
    
    \begin{figure}[H]
        \centering
        \begin{prooftree}
            \AxiomC{$\phi$}
            \AxiomC{$(\phi \to \psi)$}
            \RightLabel{(Modus Ponens)}
            \BinaryInfC{$\psi$}
        \end{prooftree}
        \caption{Modus Ponens. When both $\phi$ and $(\phi \to \psi)$ has been proved
            then $\psi$ can be derived.}
        \label{fig:modus_ponens}
    \end{figure}
\end{definition}

\begin{definition}[Proof]
    A \Keyword{proof} is then a \textit{sequence} of formulas
    
    \begin{equation}
        \phi_0, \phi_1, \ldots, \phi_n
    \end{equation}
    
    Such that for formulas $\phi_i$ with $i \le n$, $\phi_i$ is either:
    \begin{enumerate}
        \item An instance of an \textit{axiom}; or
        \item Derived by \textit{modus ponens} from $\phi_j$, $\phi_k$ for some $j, k < i$.
            \begin{itemize}
                \item For instance, if $\phi_k = (\phi_j \to \phi_i)$.
            \end{itemize}
    \end{enumerate}
    
    Given that such proof exists, then $\phi_n$ is a \Keyword{theorem} and may be denoted
    as
    \begin{equation}
        \vdash \phi_n
    \end{equation}
\end{definition}

\begin{example}
    Let us try to prove $\vdash (p \to p)$ with the axiom schema.
    
    \begin{align*}
        &((p \to ((p \to p) \to p)) \to ((p \to (p \to p)) \to (p \to p)))
            &\quad\text{ (1) (Axiom $\mathbf{II}$) } \\
        &(p \to ((p \to p) \to p))
            &\quad\text{ (2) (Axiom $\mathbf{I}$) } \\
        &((p \to (p \to p)) \to (p \to p))
            &\quad\text{ (3) (M.P. from (1) and (2)) } \\
        &(p \to (p \to p))
            &\quad\text{ (4) (Axiom $\mathbf{I}$) } \\
        &(p \to p)
            &\quad\text{ (5) (M.P. from (3) and (4)) }
    \end{align*}
\end{example}

\begin{definition}[Axiom Schemas for Propositional Logic]\label{axi_sch:prop_logic}
    To extend \textsc{ImpNeg} to cover propositional language, we add new axiom schemas
    to that of Definition \ref{axi_sch:prop_logic} to handle \textit{double negation}, 
    \textit{disjunction} and \textit{conjunction}.
    
    \begin{enumerate}
        \item \Keyword{IV}: $(p \to \neg \neg p)$ and $(\neg \neg p \to q)$
        \item \Keyword{V}: $((p \lor q) \Iff (\neg p \to q))$
        \item \Keyword{VI}: $((p \land q) \Iff \neg(p \to \neg q)$
    \end{enumerate}
\end{definition}

\begin{definition}[Proof with Assumptions]
    Let $\Gamma$ denote the set of \Keyword{assumptions}.
    
    Then the existence of a proof of $\phi$ using \textit{assumptions} from $\Gamma$
    is denoted as
    
    \begin{equation}
        \Gamma \vdash \phi
    \end{equation}
    
    In particular, there is a sequence
    
    \begin{equation}
        \phi_0, \phi_1, \ldots, \phi_n    
    \end{equation}
    
    Where $\phi = \phi_n$ and each of $\phi_i$ given $i \le n$ is either:
    
    \begin{enumerate}
        \item An instance of an \textit{axiom schema}; or
        \item An \textit{assumption} $\phi_i \in \Gamma$; or
        \item Obtained from $\phi_j, \phi_k$ given $j, k < i$ via \textit{modus ponens}.
    \end{enumerate}
\end{definition}

\subsection{Soundness, Completeness and Termination}

To check \Keyword{soundness} of axiomatic proof systems, we check that:

\begin{enumerate}
    \item All \textit{axioms} are \textit{valid}.
        \begin{itemize}
            \item Check that the following holds:
                \begin{prooftree}
                    \AxiomC{$\phi$}
                    \AxiomC{$(\phi \to \psi)$}
                    \RightLabel{(Modus Ponens)}
                    \BinaryInfC{$\psi$}
                \end{prooftree}
        \end{itemize}
    \item Deduce that all \textit{provable} formulas are valid by \textit{induction}
        on the \textit{length} of proofs, such that
        \begin{equation}
            \vdash \phi \implies \models \phi
        \end{equation}
\end{enumerate}

To check \Keyword{completeness} for axiomatic proof systems, we need to check
that:

\begin{equation}
    \models \phi \implies \vdash \phi
\end{equation}

\begin{remark}
    Note that axiomatic proof systems are \textit{not} guaranteed to \textit{terminate}.
\end{remark}

\section{Propositional Tableaux}

\begin{definition}[Tableau]
    A \Keyword{tableau} can determine the \textit{satisfiability} of formula $\phi$,
    by
    
    \begin{itemize}
        \item \textit{Decomposing} the formula using rules until only
            \textit{literals} remain.
        \item A tableau will \textit{close} or not depending on the
            nature of the \textit{literals}.
        \begin{itemize}
            \item If a tableau \textit{closes} then $\phi$ is \textit{unsatisfiable}.
            \item If a tableau remains \textit{open} then $\phi$ is
                \textit{satisfiable}.
        \end{itemize}
    \end{itemize}
    
    A \Keyword{tableau} $T$ is a \textit{binary tree} where every node is labelled
    by a formula.
    
    \begin{itemize}
        \item Initially, $T$ has only the root node labelled by the target formula
            $\phi$ for which we are testing satisfiability.
        \item Every non-literal formula in $T$ gets \textit{expanded} and
            \textit{ticked}.
        \item If a \textit{branch} of $T$ contains both the literal $\psi$ 
            \textit{and} its negation $\neg \psi$ then the branch \textit{closes};
            otherwise, the branch remains \textit{open}.
        \item If \textit{every} branch of $T$ is \textit{closed} then $T$
            \textit{closes}; otherwise, $T$ remains \textit{open}.
    \end{itemize}
\end{definition}

\subsection{Tableau Expansion Rules}

There are two types of expansion rules for Propositional Tableaux.

\begin{definition}[$\alpha$-Formula Expansion]
    $\alpha$-formulas are formulas with \textit{conjunction} as the main connective,
    either directly or by expansion:
    
    \begin{enumerate}
        \item $(\phi \land \psi)$
            \begin{equation}
                \begin{forest}
                    [$(\phi \land \psi) \checkmark$
                        [$\phi$
                            [$\psi$]
                        ]
                    ]
                \end{forest}
            \end{equation}
            $(\phi \land \psi)$ is true iff $\phi$ and $\psi$ are both true.
            We add new nodes labelled by $\phi$ and $\psi$ respectively at
            \textit{every leaf} in $T$ that is \textit{below} the current node
            $(\phi \land \psi)$.
        \item $\neg \neg \phi$
            \begin{equation}
                \begin{forest}
                    [$(\neg \neg \phi) \checkmark$
                        [$\phi$]
                    ]
                \end{forest}
            \end{equation}
        \item $\neg (\phi \lor \psi)$
            \begin{equation}
                \begin{forest}
                    [$\neg (\phi \land \psi) \checkmark$
                        [$\neg \phi$
                            [$\neg \psi$]
                        ]
                    ]
                \end{forest}
            \end{equation}
            De Morgan's law. $\neg (\phi \lor \psi) \equiv \neg \phi \land \neg \psi$.
        \item $\neg (\phi \to \psi)$
            \begin{equation}
                \begin{forest}
                    [$\neg (\phi \to \psi) \checkmark$
                        [$\phi$
                            [$\neg \psi$]
                        ]
                    ]
                \end{forest}
            \end{equation}
            $\neg (\phi \to \psi) 
                \equiv \neg (\neg \phi \lor \psi)
                \equiv \phi \land \neg \psi$.
    \end{enumerate}
\end{definition}

\begin{definition}[$\beta$-Formula Expansion]
    $\beta$-formulas are formulas with \textit{disjunction} as the main connective,
    either directly or by expansion:
    
    \begin{enumerate}
        \item $(\phi \lor \psi)$
            \begin{equation}
                \begin{forest}
                    [$(\phi \lor \psi) \checkmark$
                        [$\phi$]
                        [$\psi$]
                    ]
                \end{forest}
            \end{equation}
            $(\psi \lor \psi)$ is true if either of $\phi$ or $\psi$ is true,
            without needing both of them to be true. Hence, we create two
            branches to match such possibility. The two branches are to be
            added under \textit{every leaf} of $T$ \textit{below} the current 
            node $(\phi \lor \psi)$.
        \item $\neg (\phi \lor \psi)$
            \begin{equation}
                \begin{forest}
                    [$\neg (\phi \land \psi) \checkmark$
                        [$\neg \phi$]
                        [$\neg \psi$]
                    ]
                \end{forest}
            \end{equation}
            De Morgan's law. $\neg (\phi \land \psi)
            \equiv \neg \phi \lor \neg \psi$.
        \item $(\phi \to \psi)$
            \begin{equation}
                \begin{forest}
                    [$(\phi \to \psi) \checkmark$
                        [$\neg \phi$]
                        [$\psi$]
                    ]
                \end{forest}
            \end{equation}
            $(\phi \to \psi) \equiv \neg \phi \lor \psi$.
    \end{enumerate}
\end{definition}

\begin{definition}[Closed Tableau]
    If $\phi$ is at the \textit{root} of a \textit{closed tableau}, then
    $\phi$ is \textit{unsatisfiable}.
\end{definition}

\begin{remark}
    If $\phi$ is at the root of an open tableau, it does not necessarily
    mean that it is satisfiable; $\phi$ may not yet be full expanded
    where every node in tableau $T$ is either a literal or ticked.
    
    \begin{equation*}
        \begin{forest}
            [$((\phi \land \neg \phi) \lor (\psi \and \neg \psi)) \checkmark$
                [$(\phi \land \neg \phi)$]
                [$(\psi \and \neg \psi)$]
            ]
        \end{forest}
    \end{equation*}
    
    Note that even though $((\phi \land \neg \phi) \lor (\psi \and \neg \psi))$
    is at the root of an open tableau, because its left and right sub trees
    are not full expanded to completion, it cannot yet be considered
    satisfiable.
\end{remark}

\begin{definition}[Complete Tableau]
    A \textit{tableau} $T$ is \Keyword{complete} iff \textit{every} node is
    either:
    
    \begin{enumerate}
        \item \textit{Ticked} $\checkmark$ (\textit{expanded}); or
        \item A \textit{literal}.
    \end{enumerate}
\end{definition}

\begin{definition}[Complete Open Tableau]
    If $\phi$ is the \textit{root} of an \Keyword{complete open} tableau,
    then $\phi$ is \textit{satisfiable}.
\end{definition}

\begin{remark}
    Since a proof system is interested in \textit{validity} and not just
    satisfiability, we can exploit the negation of a formula to test for
    validity:
    
    \begin{align}
        \phi \Stext{satisfiable} &\WIff \neg \phi \Stext{invalid} \\
        \phi \Stext{valid} &\WIff \neg \phi \Stext{unsatisfiable}
    \end{align}
    
    Hence, we can test the validity of $\phi$ by checking if the tableau
    for $\neg \phi$ closes. If the tableau for $\neg \phi$ closes,
    then $\neg \phi$ is \textit{unsatisfiable} and thus $\phi$ is 
    \textit{valid}.
    
    We can use $\bigoplus$ to denote that a branch \textit{closes}.
\end{remark}

\subsection{Tableau and Disjunctive Normal Form (DNF)}

\begin{definition}[Disjunctive Normal Form (DNF)]
    A formula is in \Keyword{disjunctive normal form} (\textbf{DNF}) iff
    it is a \textit{disjunction} of one or more clauses, where each
    clause is a \textit{conjunction} of one or more literals.
    
    \begin{figure}[H]
        \centering
        \begin{grammar}
        <disjunction> ::= \lit{$($} <conjunction> \lit{$\lor$} 
                <disjunction> \lit{$)$}
            \alt <conjunction>
        
        <conjunction> ::= \lit{$($} <literal> \lit{$\lor$} 
                <conjunction> \lit{$)$}
            \alt <literal>
        
        <literal> ::= \lit{$\neg$} $v$ 
                \quad\text{// where $v \in \mathcal{V}$}
            \alt $v$
                \quad\text{// where $v \in \mathcal{V}$}
        \end{grammar}
        \caption{BNF Grammar for Disjunctive Normal Form (DNF).}
        \label{fig:dnf_bnf}
    \end{figure}
\end{definition}

\begin{definition}[Tableau to DNF Conversion]
    We can use a \textit{tableau} to find a \textit{DNF equivalent} for
    some formula $\phi$.
    
    \begin{enumerate}
        \item Place $\phi$ at \textit{root} of a new tableau $T$.
        \item \textit{Completely expand} $T$.
        \item For \textit{every open} branch $\Theta_i$ in $T$, let
            clause $\mathbf{C}_i$ be defined as the conjunction of the 
            $m$ literals $\mathbf{L}_i = \Set{ l_1, \ldots, l_m }$ in 
            $\Theta_i$
            \begin{equation}
                \mathbf{C}_i = \bigwedge \mathbf{L}_i
            \end{equation}
        \item Then the DNF equivalent $\phi'$ of $\phi$ is simply the 
        disjunction of all the clauses 
        $\mathbf{C} = \Set{ \mathbf{C}_1, \ldots, \mathbf{C}_n }$
        formed from the $n$ open branches $\Theta_1, \ldots, \Theta_n$.
            \begin{equation}
                \phi' \equiv \bigvee \mathbf{C}
            \end{equation}
    \end{enumerate}
\end{definition}
