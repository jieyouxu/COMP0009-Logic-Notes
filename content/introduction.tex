\chapter{Introduction}

\section{Formal Logic}

Formal logic consists of three parts:

\begin{enumerate}
    \item \Keyword{Syntax}: the grammar of the logic language.
    \item \Keyword{Semantics}: how the language is to be interpreted.
    \item \Keyword{Inference} or \Keyword{Proof System}: how true statements are
        to be proved through reasoning.
\end{enumerate}

\section{Propositional Logic}

\subsection{Grammar}

The grammar for \Keyword{Propositional logic} is given by Figure \ref{fig:propositional_logic_bnf}.

\begin{figure}[H]
    \centering
    \begin{grammar}
    <formula> ::= $($ <formula> <binary-op> <formula> $)$
        \alt $\neg$ <formula>
        \alt <proposition>
    
    <binary-op> ::= $\land$ | $\lor$ | $\to$
    
    <proposition> ::= $p$ | $q$ | $r$ | $\cdots$
    \end{grammar}
    \caption{BNF Grammar for Propositional Logic.}
    \label{fig:propositional_logic_bnf}
\end{figure}

Each $\Angled{\textit{proposition}}$ is a variable from a set of
\Keyword{variables} $\mathcal{V}$ that is assumed to be infinite. That is, $p, q, r \in \mathcal{V}$.

The \textit{binary operators} ($\Angled{\textit{binary-op}}$) $\land$, $\lor$ and $\to$ are also called \Keyword{connectives}.

\begin{definition}[Literal]
    A \Keyword{literal} is either a \textit{proposition} or 
    its \textit{negation}.
\end{definition}

\begin{example}[Literal]
    $p$ and $\neg q$ are both literals but $(p \land q)$ is not.
\end{example}

\begin{definition}[Main Connective].
    The \Keyword{main connective} of a propositional formula $\phi$ is defined to be the connective with the \textit{largest scope}.
\end{definition}

\begin{example}[Main Connective]
    Let
    \begin{equation*}
        \phi \DefAs ((p \land q) \lor (q \to r))
    \end{equation*}
    
    Then $\lor$ is the \textit{main connective} since it has the largest scope.
\end{example}

\subsection{Semantics}

To interpret propositional formulas, we need \Keyword{valuations} to map propositions to $\Set{ \top, \bot }$ (\textit{true} or \textit{false}, respectively).

\begin{definition}[Valuation]
    A \Keyword{valuation} $v$ maps any proposition $p \in \mathcal{V}$ to $\Set{ \top, \bot }$
    
    \begin{equation}
        v \colon \mathcal{V} \to \Set{ \top, \bot }
    \end{equation}
    
    $v$ satisfies the properties:
    
    \begin{align}
        v(\neg \phi) = \top &\WIff v(\phi) = \bot \\
        v(\phi \land \psi) = \top &\WIff v(\phi) = v(\psi) = \top \\
        v(\phi \lor \psi) = \top &\WIff v(\phi) = \top \Stext{or} v(\psi) = \top \\
        v(\phi \to \psi) = \top &\WIff = v(\phi) = \bot \Stext{or} v(\psi) = \top
    \end{align}
\end{definition}

\subsection{Validity, Satisfiability, Equivalence}

\begin{definition}[Valid]
    A formula $\phi$ is \Keyword{valid} iff
    
    \begin{equation}
        \Forall v \in \text{Valuations} \colon v(\phi) = \top
    \end{equation}
    
    That is, $\phi$ must evaluate to true when interpreted by \textit{any} valuation $v \colon \mathcal{V} \to \Set{\top, \bot}$.
\end{definition}

\begin{definition}[Satisfiable]
    A formula $\phi$ is \Keyword{satisfiable} iff
    
    \begin{equation}
        \Exists v \in \text{Valuations} \colon v(\phi) = \top
    \end{equation}
    
    That is, $\phi$ must evaluate to true for \textit{at least one} valuation $v$.
\end{definition}

\begin{definition}[Equivalent]
    The formulas $\phi$ and $\psi$ are \Keyword{logically equivalent}, denoted $\phi \equiv \psi$, iff
    
    \begin{equation}
        \Forall v \in \text{Valuations} \colon v(\phi) = v(\psi)
    \end{equation}
\end{definition}

\begin{remark}
    Every \textit{valid} formula is \textit{satisfiable}, but not the converse:
    
    \begin{itemize}
        \item Validity $\to$ Satisfiability.
        \item Satisfiability $\not\to$ Validity.
    \end{itemize}
\end{remark}
