\chapter{Introduction}

\section{Formal Logic}

Formal logic consists of three parts:

\begin{enumerate}
    \item \Keyword{Syntax}: the grammar of the logic language.
    \item \Keyword{Semantics}: how the language is to be interpreted.
    \item \Keyword{Inference} or \Keyword{Proof System}: how true statements are
        to be proved through reasoning.
\end{enumerate}

\section{Propositional Logic}

\subsection{Syntax}

The grammar for \Keyword{Propositional logic} is given by Figure 
\ref{fig:propositional_logic_bnf}.

\begin{figure}[H]
    \centering
    \begin{grammar}
    <formula> ::= \lit{$($} <formula> <binary-op> <formula> \lit{$)$}
        \alt \lit{$\neg$} <formula>
        \alt <proposition>
    
    <binary-op> ::= \lit{$\land$} | \lit{$\lor$} | \lit{$\to$} | \lit{$\Iff$}
    
    <proposition> ::= \lit{$p$} | \lit{$q$} | \lit{$r$} | $\cdots$
    \end{grammar}
    \caption{BNF Grammar for Propositional Logic.}
    \label{fig:propositional_logic_bnf}
\end{figure}

Each $\Angled{\textit{proposition}}$ is a variable from a set of
\Keyword{variables} $\mathcal{V}$ that is assumed to be infinite. That is, 
$p, q, r \in \mathcal{V}$.

The \textit{binary operators} ($\Angled{\textit{binary-op}}$) $\land$, $\lor$ 
and $\to$ are also called \Keyword{connectives}.

\begin{definition}[Literal]
    A \Keyword{literal} is either a \textit{proposition} or 
    its \textit{negation}.
\end{definition}

\begin{example}[Literal]
    $p$ and $\neg q$ are both literals but $(p \land q)$ is not.
\end{example}

\begin{definition}[Main Connective].
    The \Keyword{main connective} of a propositional formula $\phi$ is defined 
    to be the connective with the \textit{largest scope}.
\end{definition}

\begin{example}[Main Connective]
    Let
    \begin{equation*}
        \phi \DefAs ((p \land q) \lor (q \to r))
    \end{equation*}
    
    Then $\lor$ is the \textit{main connective} since it has the largest scope.
\end{example}

\subsection{Semantics}

To interpret propositional formulas, we need \Keyword{valuations} to map 
propositions to $\Set{ \top, \bot }$ (\textit{true} or \textit{false}, 
respectively).

\begin{definition}[Valuation]
    A \Keyword{valuation} $v$ maps any proposition $p \in \mathcal{V}$ 
    to $\Set{ \top, \bot }$
    
    \begin{equation}
        v \colon \mathcal{V} \to \Set{ \top, \bot }
    \end{equation}
    
    $v$ satisfies the properties:
    
    \begin{align}
        v(\neg \phi) = \top &\WIff v(\phi) = \bot \\
        v(\phi \land \psi) = \top &\WIff v(\phi) = v(\psi) = \top \\
        v(\phi \lor \psi) = \top 
            &\WIff v(\phi) = \top \Stext{or} v(\psi) = \top \\
        v(\phi \to \psi) = \top 
            &\WIff = v(\phi) = \bot \Stext{or} v(\psi) = \top
    \end{align}
\end{definition}

\subsection{Validity, Satisfiability and Equivalence}

\begin{definition}[Valid]
    A formula $\phi$ is \Keyword{valid} iff
    
    \begin{equation}
        \Forall v \in \text{Valuations} \colon v(\phi) = \top
    \end{equation}
    
    That is, $\phi$ must evaluate to true when interpreted by \textit{any} 
    valuation $v \colon \mathcal{V} \to \Set{\top, \bot}$.
\end{definition}

\begin{definition}[Satisfiable]
    A formula $\phi$ is \Keyword{satisfiable} iff
    
    \begin{equation}
        \Exists v \in \text{Valuations} \colon v(\phi) = \top
    \end{equation}
    
    That is, $\phi$ must evaluate to true for \textit{at least one} valuation 
    $v$.
\end{definition}

\begin{definition}[Equivalent]
    The formulas $\phi$ and $\psi$ are \Keyword{logically equivalent}, denoted 
    $\phi \equiv \psi$, iff
    
    \begin{equation}
        \Forall v \in \text{Valuations} \colon v(\phi) = v(\psi)
    \end{equation}
\end{definition}

\begin{remark}
    Every \textit{valid} formula is \textit{satisfiable}, but not the converse:
    
    \begin{itemize}
        \item Validity $\to$ Satisfiability.
        \item Satisfiability $\not\to$ Validity.
    \end{itemize}
\end{remark}

\section{First-Order Logic (Predicate Logic)}

\subsection{Syntax}

\begin{definition}[First-Order Logic]
    A \Keyword{first-order logic} language $L$ is a 3-tuple

    \begin{equation}
        L \DefAs \Angled{\mathcal{C}, \mathcal{F}, \mathcal{P}}
    \end{equation}

    Where:

    \begin{enumerate}
        \item $\mathcal{C}$ is the set of \Keyword{constant symbols}.
        \item $\mathcal{F}$ is the set of \Keyword{function symbols}.
            Let $f^n$ denote function $f$ is of \Keyword{arity} $n$.
        \item $\mathcal{P}$ is the non-empty set of \Keyword{predicate symbols}.
            Let $p^n$ denote predicate $p$ is of \Keyword{arity} $n$.
    \end{enumerate}
    
    And let $\mathcal{V}$ be the infinite set of \textit{variable symbols}.
\end{definition}

The grammar of first-order logic is given in Figure 
\ref{fig:first_order_logic_bnf}.

\begin{figure}[H]
    \centering
    \begin{grammar}
    <formula> ::= \lit{$($} <formula>$_0$ <binary-op> <formula>$_1$ \lit{$)$}
        \alt \lit{$\Forall$} $v$ \lit{$\colon$} <formula>
            \quad\text{// where $v \in \mathcal{V}$}
        \alt \lit{$\Exists$} $v$ \lit{$\colon$} <formula>
            \quad\text{// where $v \in \mathcal{V}$}
        \alt \lit{$\neg$} <formula>
        \alt <atom>
    
    <binary-op> ::= \lit{$\land$} | \lit{$\lor$} | \lit{$\to$} | \lit{$\Iff$}
    
    <atom> ::= $p^n$ \lit{$($} <term>$_0$ \lit{,} <term>$_1$ \lit{,} 
        $\ldots$ \lit{,} <term>$_{n-1}$ \lit{$)$}
            \quad\text{// where $p^n \in \mathcal{P}$}
    
    <term> ::= $f^n$ \lit{$($} <term>$_0$ \lit{,} <term>$_1$ \lit{,} 
            $\ldots$ \lit{,} <term>$_{n-1}$ \lit{$)$}
                \quad\text{// where $f^n \in \mathcal{F}$}
        \alt $c$ \quad\text{// where $c \in \mathcal{C}$}
        \alt $v$ \quad\text{// where $v \in \mathcal{V}$}
    \end{grammar}
    \caption{BNF Grammar for First-Order Logic.}
    \label{fig:first_order_logic_bnf}
\end{figure}

\begin{definition}[Arity]
    The \Keyword{arity} of a \textit{predicate symbol} or a 
    \textit{function symbol} is the \textit{number of arguments} that the 
    predicate or function symbol takes.
    
    \textit{Constant symbols} are of arity $0$ since they do not take arguments.
\end{definition}

\begin{definition}[Predicate Formula]
    There are two types of \textit{predicate formula}, as specified in the BNF 
    of Figure \ref{fig:first_order_logic_bnf}:
    
    \begin{enumerate}
        \item \Keyword{Atomic} formula, of the form
            \begin{equation}
                R(\tau_0, \ldots, \tau_{n-1})
            \end{equation}
            
            Where
            \begin{itemize}
                \item $\tau_{i}$ is some \textit{term} given $i < n$; and
                \item $R \in \mathcal{P}$ is a \textit{predicate} of arity $n$.
            \end{itemize}
        \item \Keyword{Formula}, of the form
            \begin{equation}
                \phi \Coloneqq \mathrm{Atom} 
                    \mid \neg \phi 
                    \mid ( \phi \neg \phi' )
                    \mid \Exists x \colon \phi
            \end{equation}
            
            And we can define
            $\Forall x \colon \phi\equiv \neg \Exists x \colon \neg \phi$.
    \end{enumerate}
\end{definition}

\subsection{Semantics}

\begin{definition}[$L$-Structure]
    A \Keyword{first-order structure} or a \Keyword{$L$-structure} is a 2-tuple
    
    \begin{equation}
        L\text{-structure} \DefAs \Angled{D, \mathcal{I}}
    \end{equation}
    
    Where
    \begin{enumerate}
        \item $D$ is the \Keyword{domain}, a non-empty set; and
        \item $\mathcal{I}$ is the valuation function which interprets
            \textit{constants}, \textit{functions} and \textit{predicates}.
    \end{enumerate}
\end{definition}

\begin{definition}[Interpretation]
    The type of the \Keyword{interpretation} $\mathcal{I}$ which interprets
    \textit{constants}, \textit{functions} and \textit{predicates} is given as
    
    \begin{align}
        \mathcal{I} \colon 
            &\ \mathcal{C} \to D \\
        \mathcal{I}(f) \colon 
            &\ D^n \to D
                \quad\text{(iff $f \in \mathcal{F}$ is a $n$-ary function)} \\
        \mathcal{I}(R) \subseteq 
            &\ D^n \quad\text{(iff $p \in \mathcal{R}$ is a $n$-ary predicate)}    
    \end{align}
\end{definition}

\begin{definition}[Variable Assignment]
    We define the \Keyword{variable assignment} function
    
    \begin{equation}
        \mathcal{A} \colon \mathcal{V} \to D
    \end{equation}
    
    So that we can associate each \textit{variable} $v \in \mathcal{V}$ with 
    a corresponding value $d \in \mathcal{D}$ of the \textit{domain}.
\end{definition}

\begin{definition}[Evaluating Terms]
    Let $\mathcal{S} = \Angled{D, \mathcal{I}}$ be a $L$-structure and 
    $\mathcal{A} \colon \mathcal{V} \to D$ be variable assignment. We denote the
    evaluation of first-order term $\tau$ with
    
    \begin{equation}
        [\tau]^{\mathcal{S}, \mathcal{A}}
    \end{equation}
    
    To mean term $\tau$ is evaluated in the $L$-structure $\mathcal{S}$ with 
    variable assignments given by $\mathcal{A}$. It is inductively defined as
    
    \begin{align}
        [c]^{\mathcal{S}, \mathcal{A}} 
            &= I(c) \quad\text{(given $c \in \mathcal{C}$)} \\
        [x]^{\mathcal{S}, \mathcal{A}} 
            &= A(x) \quad\text{(given $x \in \mathcal{V}$)} \\
        [f(\tau_0, \ldots, \tau_{n-1})]^{\mathcal{S}, \mathcal{A}} 
            &= I(f)([\tau_0]^{\mathcal{S}, \mathcal{A}}, \ldots, 
                [\tau_{n-1}]^{\mathcal{S}, \mathcal{A}})
    \end{align}
\end{definition}

\begin{remark}
    Let $\mathcal{S}, \mathcal{A} \models \phi$ denote the interpretation of 
    formula $\phi$ under $L$-structure $\mathcal{S}$ with variable assignment 
    $\mathcal{A}$.
    
    Let
    
    \begin{equation*}
        \mathcal{S}, \mathcal{A} \models \phi
            \WIff \mathcal{S}, \mathcal{A} \models \psi
    \end{equation*}
    
    Denote that $\mathcal{S}, \mathcal{A} \models \phi$ is valid if and only if
    $\mathcal{S}, \mathcal{A} \models \psi$ is valid.
\end{remark}

\begin{definition}[Evaluating Formulas]
    Then formulas can be evaluated inductively, with
    
    \begin{align}
        \mathcal{S}, \mathcal{A} \models \neg \phi 
            &\WIff \mathcal{S}, \mathcal{A} \not\models \phi \\
        \mathcal{S}, \mathcal{A} \models (\phi \land \psi)
            &\WIff \mathcal{S}, \mathcal{A} \models \phi
                \Stext{and} \mathcal{S}, \mathcal{A} \models \phi \\
        \mathcal{S}, \mathcal{A} \models (\phi \lor \psi)
            &\WIff \mathcal{S}, \mathcal{A} \models \phi
                \Stext{or} \mathcal{S}, \mathcal{A} \models \phi \\
        \mathcal{S}, \mathcal{A} \models R(\tau_0, \dots, \tau_{n-1})
            &\WIff \left(
                [\tau_0]^{\mathcal{S}, \mathcal{A}},
                \ldots, 
                [\tau_{n-1}]^{\mathcal{S}, \mathcal{A}} 
            \right) \in \mathcal{I}(R) \quad\text{(iff $R \in \mathcal{P}$)} \\
        \mathcal{S}, \mathcal{A} \models \Exists \phi
            &\WIff \mathcal{S}, \mathcal{A}[x \mapsto d] \models \phi 
            \Stext{for some} d \in D \\
        \mathcal{S}, \mathcal{A} \models \Forall \phi
            &\WIff \mathcal{S}, \mathcal{A}[x \mapsto d] \models \phi
            \Stext{for all} d \in D
    \end{align}
\end{definition}

\subsection{Validity and Satisfiability}

\begin{definition}[Validity]
    Let $\mathcal{S} \DefAs \Angled{D, \mathcal{I}}$ be an $L$-structure 
    and $\phi$ be a formula. Then
    
    \begin{itemize}
        \item $\mathcal{S} \models \phi$ \quad means $\phi$ is
            \Keyword{valid in structure $\mathcal{S}$} iff:
            \begin{itemize}
                \item For \textit{all} assignments $\mathcal{A} \colon \mathcal{V} \to D$,
                    we have $\mathcal{S}, \mathcal{A} \models \phi$.
            \end{itemize}
        \item $\models \phi$ \quad means $\phi$ is \Keyword{valid} iff:
            \begin{itemize}
                \item For \textit{all} $L$-structures $S$, we have 
                    $\mathcal{S} \models \phi$.
            \end{itemize}
    \end{itemize}
\end{definition}

\begin{definition}[Satisfiability]
    Let $\mathcal{S} \DefAs \Angled{D, \mathcal{I}}$ be an $L$-structure 
    and $\phi$ be a formula. Then
    
    \begin{itemize}
        \item $\phi$ is \Keyword{satisfiable in $\mathcal{S}$} iff:
            \begin{itemize}
                \item There is \textit{some} assignment 
                    $\mathcal{A} \colon \mathcal{V} \to D$ making 
                    $\mathcal{S}, \mathcal{A} \models \phi$ valid.
            \end{itemize}
        \item $\phi$ is \Keyword{satisfiable} iff:
            \begin{itemize}
                \item There is \textit{some} $L$-structure $\mathcal{S}$ in 
                    which it is \textit{satisfiable}.
            \end{itemize}
    \end{itemize}
\end{definition}

\begin{remark}
    $\phi$ is \Keyword{invalid} $\WIff$ $\neg \phi$ is \Keyword{satisfiable}.
\end{remark}
